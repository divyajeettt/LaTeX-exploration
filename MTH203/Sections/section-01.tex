\section{Multivariable Functions}

\subsection{Functions in two variables}
Let $X \subseteq \R$ and $Y \subseteq \R$. If for every $(x, y) \in X \times Y, \exists \text{ a unique } z \in \R$
according to some rule $z = f(x, y)$, then $f(x, y)$ is said to be a real-valued function in two variables, $x$ and $y$, given by
\begin{equation}
    z = f(x, y),\ (x, y) \in X \times Y
\end{equation}

The variables $x$ and $y$ are $independent$ and $z$ is $dependent$ (on $x$ and $y$).


\subsection{Functions in \texorpdfstring{$n$}{n} variables}
In general, let $T \in \R^n$. A real valued function $f$ on $T$ is a rule that assigns a $w \in \R$ such that
\begin{equation}
    w = f(x_1, x_2, x_3, \hdots, x_n),\ x_i \in \R\ \forall i \in \{1, 2, 3, \hdots, n\}
\end{equation}

for all elements in $T$. Here, the variables $x_1, x_2, x_3, \hdots, x_n$ are $independent$ and $w$ is $dependent$.