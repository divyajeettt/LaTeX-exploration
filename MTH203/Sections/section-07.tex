\section{Directional Derivatives and Gradients}

\subsection{Directional Derivatives}

The partial derivatives $f_x$ and $f_y$ of a function $z = f(x, y)$ are the rates of change of the output $z$ of along the
paths holding $y$ and $x$ constant, respectively. However, a derivative of a function can be found along any path.
The derivative of a function along a path other than one parallel to the axes is called the \textbf{directional derivative}.

The direction along which the derivative is calculated, is commonly notated by $\vec{u} = \langle u_1, u_2 \rangle$.

Formally, the derivative of a function $f(x, y)$ at a point $P_0 = \point$ in the direction of the unit vector
$\hat{u} = \langle u_1, u_2 \rangle$ is the number
\begin{equation}
    \left( \frac{df}{ds} \right)_{\hat{u}, P_0} = \ddvp = \lim_{s \to 0} \frac{f(x_0 + su_1, y_0 + su_2) - f\point}{s}
\end{equation}

provided the limit exists.

\textbf{Reference Link:}
\reference{https://www.youtube.com/watch?v=GJODOGq7cAY}{Directional Derivatives: What's the slope in any direction?}


\subsection{Gradients}

\subsubsection{Gradient of a function in two variables}
The gradient (gradient) of a function $f(x, y)$ is the vector $\langle f_x, f_y \rangle$.
\begin{equation}
    \grad f = \llangle \frac{\del f}{\del x}, \frac{\del f}{\del y} \rrangle = \langle f_x, f_y\rangle
\end{equation}
To evaluate the gradient at any particular point $P_0 = \point \in \R^2$, the partial derivatives need to be evaluated
at $P_0$.

\subsubsection{Gradient of a function in \texorpdfstring{$n$}{n} variables}
The gradient (gradient) of a function $f(x_1, x_2, \hdots, x_n)$ is the vector
$\langle f_{x_1}, f_{x_2}, \hdots, f_{x_n} \rangle$.
\begin{equation}
    \grad f = \llangle \frac{\del f}{\del x_1}, \frac{\del f}{\del x_2}, \hdots, \frac{\del f}{\del x_n} \rrangle
    = \langle f_x, f_y\rangle
\end{equation}
To evaluate the gradient at any particular point $P_0 = (x_{1_0}, x_{2_0}, \hdots, x_{n_0}) \in \R^n$, the partial
derivatives need to be evaluated at $P_0$.

\textbf{Reference Link:}
\reference{https://www.youtube.com/watch?v=QQPz3eXXgQI}{Geometric Meaning of the Gradient Vector}


\subsection{Relationship betwen Gradients and Directional Derivatives}
The directional derivative is essentially (just like any other derivative) the rate of change of a function. The only
For the directional derivative of a function $f(x, y)$ the input point $P_0 = \point$ is nudged along the line
parametrized by $s$.

Hence, given that $x = x_0 + su_1 \text{ and } y = y_0 + su_2,$
\begin{align*}
    \left( \frac{df}{ds} \right)_{\hat{u}, P_0} &= \frac{\del f}{\del x} \bigg|_{P_0} \frac{dx}{ds} +
    \frac{\del f}{\del y} \bigg|_{P_0} \frac{dy}{ds} \\
    &= \frac{\del f}{\del x} \bigg|_{P_0} u_1 + \frac{\del f}{\del y} \bigg|_{P_0} u_2 \\
    &= \langle f_x, f_y \rangle \cdot \langle u_1, u_2 \rangle \\
    &= (\grad f)_{P_0} \cdot \hat{u}
\end{align*}


\subsection{Directional Derivative as a Dot product}

\begin{theorem}
    [Directinal Derivative as a Dot product]
    If $f(x, y)$ is differentiable in an open region containing the point $P_0 = \point$, then
    \begin{equation}
        \left( \frac{df}{ds} \right)_{\hat{u}, P_0} = \ddvp = (\grad f)_{P_0} \cdot \hat{u}
    \end{equation}
\end{theorem}

\begin{example}
    \normalfont Find the directional derivative of $f(x, y) = x^2 \sin{2y}$ at $(1, \pi/2)$ along $\langle 3, -4 \rangle$.

    Given $P_0 = (1, \pi/2)$ and $\vec{u} = \langle 3, -4 \rangle \implies \hat{u} = \langle \frac{3}{5}, \frac{-4}{5} \rangle$
    \begin{align*}
        \grad f &= \llangle 2x \sin{2y},\ 2 x^2 \cos{2y} \rrangle \implies
        (\grad f)_{P_0} = \langle 0, 2\rangle \\
        \ddvp &= 0 \cdot \frac{3}{5} + 2 \cdot \frac{-4}{5} = 0 + \frac{-8}{5} = - \frac{8}{5}
    \end{align*}
\end{example}


\subsection{Direction of maximum, minimum, and zero change}

We can find the direction of extremum change of the function $f(x, y)$ by comparing the values of directional derivatives.
If $\theta$ is the angle between the gradient of $f$ at $P_0$ and the direction $\hat{u}$,
\begin{align*}
    \ddv &= \grad f \cdot \hat{u} \\
    \lVert \ddv \rVert &= \lVert \grad f \rVert\ \cos{\theta}
\end{align*}

And so the following three cases arise:
\begin{enumerate}
    \item $\theta = 0$: $\ddv$ is maximum, i.e. $\lVert \grad f \rVert$. $f$ increases the most rapidly
    when $\hat{u}$ is in the direction of $\grad f$.

    \item $\theta = \pi$: $\ddv$ is minimum, i.e. $- \lVert \grad f \rVert$. $f$ decreases the most rapidly
    when $\hat{u}$ is in the direction of $\grad f$.

    \item $\theta = \pi/2$: $\ddv = 0$, i.e. moving along $\hat{u}$ has no effect on $f$, i.e. the movement is along a contour.
\end{enumerate}

\begin{example}
    \normalfont Find the directions of maximum increase, maximum decrease, and zero change of
    $f(x, y) = \frac{x^2}{2} + \frac{y^2}{2}$ at the point $(1, 1)$.

    The gradient of the function at $(1, 1)$ is $(\grad f)_{(1, 1)} = \langle x, y \rangle_{(1, 1)} = \langle 1, 1\rangle$.

    \begin{enumerate}
        \item
        The function increases the most rapidly in the direction of $\grad f$. Therefore, the direction of maximum increase is
        $$\hat{u} = \llangle \frac{1}{\sqrt{2}}, \frac{1}{\sqrt{2}} \rrangle =
        \frac{1}{\sqrt{2}} \hat{i} + \frac{1}{\sqrt{2}} \hat{j}$$

        \item
        The function decreases the most rapidly in the direction opposite to $\grad f$. Therefore, the
        direction of maximum decrease is
        $$\hat{u} = \llangle -\frac{1}{\sqrt{2}}, -\frac{1}{\sqrt{2}} \rrangle =
        -\frac{1}{\sqrt{2}} \hat{i} - \frac{1}{\sqrt{2}} \hat{j}$$

        \item
        The direction of zero change are the directions orthogonal to $\grad f$. There are
        $$\hat{n} = \llangle \frac{1}{\sqrt{2}}, -\frac{1}{\sqrt{2}} \rrangle =
        \frac{1}{\sqrt{2}} \hat{i} -\frac{1}{\sqrt{2}} \hat{j} \text{ and }
        -\hat{n} = \llangle - \frac{1}{\sqrt{2}}, \frac{1}{\sqrt{2}} \rrangle =
        -\frac{1}{\sqrt{2}} \hat{i} + \frac{1}{\sqrt{2}} \hat{j}$$
    \end{enumerate}
\end{example}


\subsection{Properties of the Gradient}
The following algebraic rules are obeyed by gradients. The rules are similar to the algebra derivatives (notice the
difference in the \textbf{Quotient Rule}):

\begin{enumerate}
    \item \textbf{Scaling Rule:} $\grad (kf) = k \grad f$
    \item \textbf{Sum/Difference Rule:} $\grad (f \pm g) = \grad f \pm \grad g$
    \item \textbf{Product Rule:} $\grad (fg) = f \grad g + g \grad f$
    \item \textbf{Quotient Rule:} $\grad \left( \frac{f}{g} \right) = \frac{g\grad f - f\grad g}{g^2}$
\end{enumerate}