\documentclass{article}
\usepackage[document]{ragged2e}
\usepackage[utf8]{inputenc}
\usepackage{amsfonts, amsmath}
\usepackage{geometry}
\usepackage{enumitem}
\usepackage{parskip}
\usepackage{setspace}

%\baselineskip=20pt

\newcommand{\R}{\mathbb{R}}
\newcommand{\N}{\mathbb{N}}

\newcommand{\column}[3]{\begin{bmatrix}\ #1\ \\ \ #2\ \\ \ #3\ \end{bmatrix}}

%% In the original PDF, the top margin is 27mm
\geometry{a4paper, left=30mm, right=30mm, top=30mm}

%% Uncomment the following line to remove page numbers
% \pagenumbering{gobble}

\title{Mid-Semester Exam: Linear Algebra}
\author{Indraprastha Institute of Information Technology, Delhi}
\date{19th February, 2022}


\begin{document}
\setstretch{1.3}

\maketitle

\textbf{Duration:} 60 minutes \hfill \textbf{Maximum Marks:} 50 \\[5mm]

\textbf{Question 1.}
\begin{enumerate}[label=(\alph*), leftmargin=6.25mm]
    \item (5 marks) Let $A$ and $B$ be $m \times n$ matrices (where $m, n \in \N$) having columns $\textbf{a}_1, \ldots , \textbf{a}_n$, and $\textbf{b}_1, \ldots, \textbf{b}_n$, respectively. Suppose $\textbf{b}_j = j^2 \textbf{a}_{j-1}$ for $j = 2, \ldots, n$ and $\textbf{b}_1 = \textbf{a}_n$. Find an $ n \times n$ matrix $E$ such that $AE = B$.

    \item (5 marks) Let $A = \begin{bmatrix}
        \ 0 & 0 & 1\ \\
        \ 4 & 0 & 0\ \\
        \ 0 & 9 & 0\ \\
    \end{bmatrix}$. Let $\textbf{b}_1 = \column{1}{0}{1}$ and $\textbf{b}_2 = \column{1}{5}{7}$. Solve the equations

    $A\textbf{x} = \textbf{b}_1$ and $A\textbf{x} = \textbf{b}_2$ by row reducing exactly one matrix. \\[8mm]
\end{enumerate}


\textbf{Question 2.} (10 marks)
All subparts of this question carry equal marks.

In each part of this question, $V$ is a vector space and $W$ is a subset of $V$. Decide whether $W$ is a subspace of $V$. Justify your answers with a short proof or counterexample.

\begin{enumerate}[label=(\alph*), leftmargin=6.25mm]
    \item
    $V = \R(t)$, the set of $all$ polynomials in $t$ which have real coefficients (please note that the degrees of the polynomials are not bounded). \\
    $W = \{ p(t) = a_0 + \cdots + a_nt^n \ |\ a_{2k} = 0$, if $k \in \N$ and $2k \in \{ 0, \ldots, n \} \}$

    \item
    $V = \R^n$ \\
    $W = \{ (x_1, \ldots, x_n)\ |\ x_1 + \cdots + x_n \ge 0 \}$

    \item
    $V = \R^n$ \\
    $W = \{ (x_1, \ldots, x_n)\ |\ x_1^2 + \cdots + x_n^2 \ge 0 \}$

    \item
    $V = \R^\infty$, the set of all sequences indexed by $\N$ \\
    Fix $k \in \N.$ \\
    $W = \{ (a_n)\ |\ a_1 + \cdots + a_k = 0 \}$

    \item
    $V = \R^{n \times n}$, the set of all $n \times n$ matrices having real entries. \\
    $(Note: \R^{n \times n}$ is the same as $M_n(\R) )$ \\
    $W = \{ A\ |\ A$ is in reduced row echelon form$\}$
\end{enumerate}

\newpage

\textbf{Question 3.} Let $A = \begin{bmatrix}
    \ a & b\ \\
    \ c & d\ \\
\end{bmatrix}$, where $a, b, c, d \ge 0,\ a + c = b + d = 1$ and $A \neq I_2$. Let $ P = \begin{bmatrix}
    \ b &\ \ 1\ \\
    \ c & -1\ \\
\end{bmatrix}$.

\begin{enumerate}[label=(\alph*), leftmargin=6.25mm]
    \item (7 marks)
    Show that $P$ is invertible. Find $P^{-1}$ and $P^{-1}AP$.

    \item (3 marks)
    Find a formula for $A^n$. \\[8mm]
\end{enumerate}


\textbf{Question 4.} Let $f: \R \to \R$ be a function.

\begin{enumerate}[label=(\alph*), leftmargin=6.25mm]
    \item (4 marks)
    Show that: If $\exists\ c \in \R \setminus \{ 0 \}$ such that $f(x) = cx,\ \forall x \in \R$, then the graph of $f$ is a proper nontrivial subspace of $\R^2$. \\
    (The graph of a function $f: \R \to \R$ is defined as the set $\{ (x, y) \ |\ y = f(x)\}$.)

    \item (1 mark)
    What is the converse of the statement that is too be proved in part (a)?

    \item (5 marks)
    Is the converse that you stated in part (b) true? Justify with a short proof or an appropriate counterexample.
\end{enumerate}

$(Note:$ You may assume the following statement without proof - \\

Any proper nontrivial subspace of $\R^2$ is of the form Span$\{ \vec{v} \}$ where $\vec{v}$ is a non-zero vector in $\R^2$.) \\[8mm]


\textbf{Question 5.} (10 marks)
Solve ONE of the following two problems (either part (a) or part (b)).

\begin{enumerate}[label=(\alph*), leftmargin=6.25mm]
    \item Prove or disprove: \\
    If $\{ \vec{v}_1, \ldots, \vec{v}_n \}$ is a basis for a vector spave $V$, then so is $\{ \alpha_1 \vec{v}_1, \ldots,  \alpha_n \vec{v}_n \}$ where the $\alpha_i$ are non-zero scalars.

    \item Prove or disprove: \\
    If $\{ \vec{v}_1, \ldots, \vec{v}_n \}$ is a basis for a vector spave $V$, then so is $\{ \vec{v}_1, \vec{v}_1 + \vec{v}_2, \ldots,  \vec{v}_1 + \vec{v}_2 + \ldots + \vec{v}_n \}$.
\end{enumerate}


\end{document}