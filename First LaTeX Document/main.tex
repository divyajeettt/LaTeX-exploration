\documentclass{article}
\usepackage[utf8]{inputenc}
\usepackage{amsfonts}
\usepackage{amsmath}

% amsfonts package enables \mthbb
% amsmath package enables \bmatrix


\title{My First \LaTeX \ Document}
\author{Divyajeet Singh}
\date{August 17, 2022}


\begin{document}


\maketitle

\section{Introduction}

This is my very first LaTeX document. In this document, I've explored some of the fancy features of LaTeX.

\begin{enumerate}
    \item The Preamble (Header)

    The Preamble of a LaTeX document is declared at the beginning using the following syntax:

    \verb|\title{your-title-here}|

    \verb|\author{name-of-author}|

    \verb|\date{date}|

    \item \verb|\maketitle| command

    This command is usually written in LaTeX documents to display the Preamble. Omitting this line in the document will result in the Preamble not being displayed in the document.

    \item \verb|\section{your-text-here}| command

    Used to create sections in the document. To avoid the list numbering, the following syntax may be used: \verb|\section*{your-text-here}|

\end{enumerate}

\section{Writing Mathematical Formulae}

This section contains different (possibly incorrect) formulae, which are written solely to explore LaTeX Syntax.

\begin{enumerate}
    \item In-line Formula

    The most beautiful equation in math is $e^{i\pi} + 1 = 0$. It relates beautifully, some of the most important constants in math, namely, $e$, $\pi$, $0$, and $1$.

    \item Formula in a single line

    \begin{itemize}
        \item The Euler's constant $e$ has various definitions, some of which include:

        $$e = \lim_{n \to \infty} \left(1 + \frac{1}{n} \right) ^ n = \sum_{n=0}^{\infty}\frac{1}{n!} = \lim_{n \to \infty} \frac{n}{\sqrt[n]{n!}}$$

        \item Now, let's look at a continued fraction that describes the golden ratio $\varphi$:

        $$\varphi = \frac{1}{1 + \frac{1}{1 + \frac{1}{1 + \frac{1}{1 + \ldots}}}}$$

    \end{itemize}

    \item Calculus!

    \begin{itemize}
        \item The following is the derivative of a function:
        $$\frac{d}{dx} \left( \frac{sin(x^2)}{n} + cos^2(x) \right) = \frac{2x cos(x^2)}{n} - sin(2x)$$

        \item
        $\forall a, b, c \in \mathbb{R}; a \leq c \leq b,$
        $$\int_a^bf(x) dx = \int_a^cf(x) dx + \int_c^bf(x) dx$$

    \end{itemize}

    \item Linear Algebra!

    \begin{itemize}
        \item Here is a vector:
        $$\vec{v} = (v_1, v_2, ..., v_n)$$

        \item Here is the dot product of two vectors:
        $$\vec{v} \cdot \vec{w} = (v_1 w_1, v_2w_2, ..., v_nw_n)$$

        \item Here's a matrix:

        $$A = \begin{bmatrix}
            a_{11} & a_{12} & a_{13} \\
            a_{21} & a_{22} & a_{32} \\
        \end{bmatrix}$$

    \end{itemize}

\end{enumerate}

\section*{Conclusion}

I find LaTeX to be an easy to use tool. The typeface is easy to learn and grasp. I look forward to having to work in it.

\end{document}