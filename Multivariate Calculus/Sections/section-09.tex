\section{Approximation, Linearization, and Total Differentials}

\subsection{Estimating change in a Specific Direction}
To estimate how much the value of a differential function $f$ changes if we move a small distance $ds$ from a point $P_0$
in a direction $\hat{u}$, the following formula is used:

\begin{equation}
    df = (\grad f \big|_{P_0} \cdot \hat{u}) \cdot ds
\end{equation}

\begin{example}
    \normalfont Estimate how much the value of $f(x, y, z) = y \sin{x} + 2yz$ will change if the point $P = (x, y, z)$
    moves 0.1 units from $P_0(0, 1, 0)$ straight towards $P_1(2, 2, -2)$.

    First, we find out the direction of movement
    \begin{align*}
        \vec{u} &= \langle 2-0, 2-1, -2-0 \rangle = \langle 2, 1, -2 \rangle \\
        \implies \hat{u} &= \llangle \frac{2}{3}, \frac{1}{3}, -\frac{2}{3} \rrangle
    \end{align*}

    The gradient of $f$ at $P_0$ is
    $$\grad f \big|_{P_0} = \langle y \cos{x}, \sin{x} + 2z, 2y \rangle_{P_0} = \langle 1, 0, 2 \rangle$$

    Therefore, estimated change $df$ on moving $ds = 0.1$ units in the direction of $\hat{u}$ will be
    \begin{align*}
        df &= (\grad f \big|_{P-0} \cdot \hat{u}) \cdot ds \\
        &= \left( \langle 1, 0, 2 \rangle \cdot \llangle \frac{2}{3}, \frac{1}{3}, -\frac{2}{3} \rrangle \right) \times 0.1 \\
        &= -\frac{2}{3} \times 0.1 \approx -0.067 \text{ units }
    \end{align*}
\end{example}


\subsection{Linearization}

\textbf{Linearization} is a linear approximation of a function near some specific point. The linearization of a function
$f(x, y)$ at a point $\point$ where $f$ is differentiable is the function

\begin{equation}
    L(x, y) = f\point + f_x \point (x - x_0) + f_y \point (y - y_0)
\end{equation}

and the approximation $f(x, y) \approx L(x, y)$ is called the \textbf{standard linear approximation} of $f$ at $\point$.

The plane $z = L(x, y)$ is tangent to the surface $z = f(x, y)$ at the point $\point$. This is because the tangent plane is a
good approximation for the function near the point of contact. Thus, linearization of a function in two variables is a tangent
plane, and the linearization of a function in one variable is a tangent line.

\begin{example}
    \normalfont Find the linearization of $f(x, y) = x^2 - xy + \frac{y^2}{2} + 3$ at the point $(3, 2)$.

    First, we find $f(3, 2), f_x(3, 2), $ and $f_y(3, 2)$
    \begin{align*}
        f(3, 2) &= 9 - 6 + 2 + 3 = 8 \\
        f_x(3, 2) &= (2x - y)_{(3, 2)} = 6 - 2 = 4 \\
        f_y(3, 2) &= (-x + y)_{(3, 2)} = -3 + 2 = -1
    \end{align*}

    Then, the linear approximation of $f$ at $(3, 2)$ will be given by
    \begin{align*}
        L(x, y) &= 8 + 4(x - 3) - 1(y - 2) \\
        &= 4x - y - 2
    \end{align*}
\end{example}


\subsection{Error in Linearization}

The absolute error in linearization is the difference in of the functions $f(x, y)$ and $L(x, y)$, i.e.
\begin{equation}
    E(x, y) = f(x, y) - L(x, y)
\end{equation}

If $f$ has continuous first and second partial derivatives throughout the region containing $\point$, and if $M$ is an
upper bound for the values of $f_{xx}, f_{yy},$ and $f_{xy}$ in the region, then the error $E(x, y)$ satisfies the
following inequality:
\begin{equation}
    |E(x, y)| \leq \frac{1}{2} M (|x - x_0| + |y - y_0|)^2
\end{equation}

\begin{example}
    \normalfont Find an upper bound for the error in the approximation $f(x, y) \approx L(x, y)$ derived in
    \textbf{Example 9.2.1} over the rectangle $R: |x - 3| \leq 0.1, |y - 2| \leq 0.1$, as a percentage of $f(3, 2)$.

    We begin by calculating $f_{xx}, f_{yy},$ and $f_{xy}$
    $$|f_{xx}| = |2| = 2, \ \ \ |f_{yy}| = |-1| = 1, \ \ \ |f_{xy}| = |1| = 1$$

    Therefore, $M$ can be assigned $\max{\{f_{xx}, f_{yy}, f_{xy}\}} = 2$. Then, given that $|x - 3| \leq 0.1$ and
    $|y - 2| \leq 0.1$,
    \begin{align*}
        |E(x, y)| &\leq \frac{1}{2} (2) (|x - 3| + |y - 2|)^2 \\
        &\leq (0.1 + 0.1) ^ 2 = 0.04
    \end{align*}

    Hence, the upper bound for the error in linearization is $0.04$, and as a percentage of $f(3, 2)$, is
    $$\frac{0.04}{8} \times 100 = 5 \%$$
\end{example}


\subsection{Total Differentials}
The \textbf{differential} of a function is the change in its value for very little changes in its inputs. So, if we move
from a point $\point$ to a point $(x_0 + dx, y_0 + dy)$, the resulting change (differential) is
\begin{equation}
    df = \Delta f \big|_{\point} = f_x \point dx + f_y \point dy
\end{equation}

This change $df$ in the linearization of $f$ is called its \textbf{total differential}.

\begin{example}
    \normalfont Suppose that a cylinder is designed to have a radius of 1 in. and a height of 5 in., but that the radius
    and height are off by $dr = +0.03$ and $dh = -0.1$. Estimate the resulting change in volume.

    For the volume of a cylinder, we have
    \begin{align*}
        V &= \pi r^2 h \\
        \Delta V \approx dV &= V_r(r_0, h_0) dr + V_h(r_0, h_0) dh
    \end{align*}

    Note that $V_r = 2 \pi r h$ and $V_h = \pi r^2$. Then
    \begin{align*}
        dV &= 2 \pi \cdot 1 \cdot 5 (0.03) + \pi \cdot 1^2 (-0.1) \\
        &= 0.3 \pi - 0.1 \pi = 0.2 \pi \approx 0.63 \text{ in.}^3
    \end{align*}
\end{example}

\begin{example}
    \normalfont The percentage errors in measured radius and height of a cylinder are no more than 2\% and 0.5\%
    respectively. Estimate the percentage error in calculation of its volume.

    Given that
    $$\left| \frac{dr}{r} \times 100 \right| \leq 2 \ \ \ \text{ and } \ \ \
    \left| \frac{dh}{h} \times 100 \right| \leq 0.5$$

    Then,
    \begin{align*}
        \frac{dV}{V} &= \frac{2 \pi r h\ dr + \pi r^2\ dh}{\pi r^2 h} = \frac{2dr}{r} + \frac{dh}{h} \\
        \left| \frac{dV}{V} \right| &\leq 2 \left| \frac{dr}{r} \right| + \left| \frac{dh}{h} \right| \\
        &\leq 2 (0.02) + 0.005 = 0.045
    \end{align*}
\end{example}


\subsection{Extension to three dimensions}
Analogous to two dimensions, the following hold in three dimensions:

\begin{enumerate}
    \item
    The linearization of a function $f(x, y, z)$ at a point $P_0 = (x_0, y_0, z_0)$ is given by
    \begin{equation}
        L(x, y, z) = f(P_0) + f_x(P_0)(x - x_0) + f_y(P_0)(y - y_0) + f_z(P_0)(z - z_0)
    \end{equation}

    \item
    Let $R$ is a rectangular region centered at $P_0$, where the partial derivates of $f$ are continuous. Suppose throughout
    $R$, the magnitude of all second derivates of $f$ are less than or equal to $M$. Then, the error
    $E(x, y, z) = f(x, y, z) - L(x, y, z)$ in the approximation of $f$ is
    \begin{equation}
        |E| \leq \frac{1}{2}M (|x - x_0| + |y - y_0| + |z - z_0|)^2
    \end{equation}

    \item
    If the second partial derivatives of $f$ are are continuous, and if $x, y,$ and $z$ change from
    $x_0, y_0,$ and $z_0$ by small amounts $dx, dy,$ and $dz$, the total differential (estimated change) is
    \begin{equation}
        df = f_x(P_0) dx + f_y(P_0) dy + f_z(P_0) dz
    \end{equation}
\end{enumerate}